\documentclass[a4paper,10pt,DIV=14]{scrartcl}
\usepackage{graphicx}
\usepackage[utf8]{inputenc} % korrekte Darstellung von Umlauten u. Sonderzeichen
\usepackage[ngerman]{babel} % Sprachpaket, ngerman = neue deutsche Rechtschreibung
\usepackage{amsmath} % Setzen mathematischer Formeln
\usepackage{titlesec}
\usepackage{float}
\usepackage{caption}
\usepackage{fancyvrb}
\usepackage{siunitx}
\usepackage{booktabs}
\usepackage{enumitem}

\usepackage{tabularx}
\newcolumntype{L}[1]{>{\raggedright\arraybackslash}p{#1}} % linksbündig mit Breitenangabe
\newcolumntype{C}[1]{>{\centering\arraybackslash}p{#1}} % zentriert mit Breitenangabe
\newcolumntype{R}[1]{>{\raggedleft\arraybackslash}p{#1}} % rechtsbündig mit Breitenangabe

\newcommand{\gqq}[1]{\glqq{}#1\grqq{}}
\newcommand{\gq}[1]{\glq{}#1\grq{}}

\renewcommand{\thesection}{Aufgabe \arabic{section}:}
\renewcommand{\thesubsection}{\alph{subsection})}
\titleformat*{\subsection}{\normalfont\fontfamily{phv}\fontsize{12}{15}\selectfont}


\captionsetup[figure]{labelformat=empty}


\begin{document}

\title{Graphische Datenverarbeitung WS17/18 \\ Theorieübung 1}
\author{
  Salmah Ahmad (2880011)
  \and
  Markus Höhn (1683303)
  \and
  Tobias Mertz (2274355)
  \and
  Steven Lamarr Reynolds (1620638)
  \and
  Sascha Zenglein (2487032)
}

\maketitle

\section{Pipeline}

\subsection{Aus was besteht der Input der Pipeline?}
Der Input der Pipeline besteht aus einer gegebenen Szenenbeschreibung.


\subsection{Zum Input gehören unter anderem \gqq{Objekte}. In welcher Form sind konkrete \gqq{Objekte} im Input gegeben?}

\begin{itemize}[itemsep=0pt]
	\item (virtuelle) Kamera
	\item Dreidimensionale Objekte
	\item Lichtquellen
	\item Beleuchtungsalgorithmen
	\item Texturen
	\item ...
\end{itemize}


\subsection{Was ist der Output der Pipeline?}
Der Output der Pipeline ist ein 2D Bild der gegeben Szenenbeschreibung.


\subsection{Weshalb ist eine Pipeline die aus $n$ Abschnitten besteht (theoretisch) $n$-mal schneller als eine Pipeline mit nur einem Abschnitt?}
Bei einer Pipeline mit $n$ Abschnitten kann eine parallele Verarbeitung durchgeführt werden.


\subsection{Weshalb ist die Pipeline Geschwindigkeit vom Bottleneck abhängig? Wieso warten die anderen Pipeline-Abschnitte bis der Bottleneck-Abschnitt fertig ist?}
Der Bottleneck-Abschnitt ist der langsamste der Pipeline.


\section{Model \& View Transformation}

\subsection{Stellen Sie die Gleichung $(x, z)^T = f(u,v)$ auf, die die $u, v$ Koordinaten in das Weltkoordinatensystem transformiert. Bestimmen Sie nun die Position der Szenenobjekte bezüglich des Weltkoordinatensystems.}


\subsection{Bestimmen Sie, welche Translation und welche Rotation auf die Szene ausgeübt werden müssen, um die Kamera in den Ursprung zu verschieben und anschließend die Blickrichtung nach -z zu rotieren.}


\subsection{Berechnen Sie die Position der Szenenobjekte nach der Model- und View-Transformation. Fertigen Sie eine Skizze an.}



\section{Optische Triangulation}

\subsection{Stellen Sie die Gleichung $\beta = f_1(p)$ auf, um aus einer Pixelposition $p$ den Winkel $\beta$ zu berechnen. Verwenden Sie dabei die Koordinate des Pixelmittelpunktes! \\ Wie groß ist $\beta$, wenn der Laserpunkt in der Mitte von Pixel 5223 registriert wird?}

\subsection{Stellen Sie die Gleichung $\alpha = f_2(\gamma)$ auf, um aus dem Spiegelwinkel $\gamma$ den Winkel $\alpha$ zu berechnen. \\ Berechnen Sie $f_2(45^\circ)$ und $f_2(77^\circ)$.}

\subsection{Stellen Sie die Gleichung $z = f_3(\alpha, \beta)$ auf, um aus den Winkeln $\alpha$ und $\beta$ den Tiefenwert $z$ zu berechnen. \\ Welcher Tiefenwert gehört zu den Winkeln $\alpha = 15^\circ$ und $\beta = 30^\circ$?}

\subsection{Stellen Sie die Gleichung $x = f_4(\beta, z)$ auf, um aus dem Winkel $\beta$ und $z$ die $x$-Koordinate zu berechnen \\ Berechnen sie $f_4(40^\circ, 100cm)$.}

\subsection{Stellen Sie nun die Gesamtgleichung $(x,z)^T = f_5(p, \gamma)$ auf, um aus der Pixelposition $p$ und dem Winkel $\gamma$ die Koordinaten $(x, z)^T$ des abgetasteten Punktes zu berechnen.}

\subsection{Zum Spiegelwinkel $\gamma = 67^\circ$ wird ein Laserpunkt im Mittelpunkt von Pixel 5730 registriert. Welche Koordinaten hat der abgetastete Punkt mit oben beschriebenen Aufbau?}

\end{document}
