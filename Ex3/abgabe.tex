\documentclass[a4paper,10pt,DIV=14]{article}
\usepackage{graphicx}
\usepackage[utf8]{inputenc} % korrekte Darstellung von Umlauten u. Sonderzeichen
\usepackage[ngerman]{babel} % Sprachpaket, ngerman = neue deutsche Rechtschreibung
\usepackage{amsmath} % Setzen mathematischer Formeln
\usepackage{amsfonts} %mathbb
\usepackage{titlesec}
\usepackage{float}
\usepackage{caption}
\usepackage{fancyvrb}
\usepackage{siunitx}
\usepackage{booktabs}
\usepackage{enumitem}
\usepackage{tikz}

\usepackage{tabularx}
\newcolumntype{L}[1]{>{\raggedright\arraybackslash}p{#1}} % linksbündig mit Breitenangabe
\newcolumntype{C}[1]{>{\centering\arraybackslash}p{#1}} % zentriert mit Breitenangabe
\newcolumntype{R}[1]{>{\raggedleft\arraybackslash}p{#1}} % rechtsbündig mit Breitenangabe

\newcommand{\gqq}[1]{\glqq{}#1\grqq{}}
\newcommand{\gq}[1]{\glq{}#1\grq{}}
\newcommand{\dg}[1]{#1^\circ}

\renewcommand{\thesection}{Aufgabe \arabic{section}:}
\renewcommand{\thesubsection}{\alph{subsection})}

\titleformat{\subsection}
{\normalsize}{\thesubsection}{1em}{}


\captionsetup[figure]{labelformat=empty}

\begin{document}

\title{Graphische Datenverarbeitung WS17/18 \\ Theorieübung 3}
\author{
  Salmah Ahmad (2880011)
  \and
  Markus Höhn (1683303)
  \and
  Tobias Mertz (2274355)
  \and
  Steven Lamarr Reynolds (1620638)
  \and
  Sascha Zenglein (2487032)
}

\maketitle

\section{Räumliche Datenstrukturen (2 Punkte)}

\begin{tikzpicture}[sibling distance=5cm, level distance=1.8cm,
level 1/.style={sibling distance=3cm},
level 2/.style={sibling distance=1.5cm},  
  every node/.style = {shape=circle, rounded corners,
    draw, align=center,
%    top color=white, bottom color=blue!20
	}]]
  \node[minimum size=1cm] {A}
    child[minimum size=1cm] { node {$B_1$} 
    	child[minimum size=1cm] { node {$C_1$}
    		child[minimum size=1cm] { node {E}}
    		child[missing, minimum size=1cm] {node {}}
    	}
    	child[minimum size=1cm] { node {F}}
    }
    child[minimum size=1cm] { node {G}
      child[minimum size=1cm] { node {H}}
      child[minimum size=1cm] { node {D}
			child[minimum size=1cm] { node {$C_2$}}     
			child[minimum size=1cm] { node {$B_2$}}
		}
      };
\end{tikzpicture}


\subsection{a) (1 Punkt)}


\subsection{b) (1 Punkt)}


\section{Projektionen (5 Punkte)}

\subsection{a)}
\subsubsection{i}
\subsubsection{ii}
\subsubsection{iii}

\subsection{b)}
\subsection{c)}
\subsection{d)}

\section{Clipping (3 Punkt)}

\end{document}
