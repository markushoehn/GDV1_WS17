\documentclass[a4paper,10pt,DIV=14]{article}
\usepackage{graphicx}
\usepackage[utf8]{inputenc} % korrekte Darstellung von Umlauten u. Sonderzeichen
\usepackage[ngerman]{babel} % Sprachpaket, ngerman = neue deutsche Rechtschreibung
\usepackage{amsmath} % Setzen mathematischer Formeln
\usepackage{amsfonts} %mathbb
\usepackage{titlesec}
\usepackage{float}
\usepackage{caption}
\usepackage{fancyvrb}
\usepackage{siunitx}
\usepackage{booktabs}
\usepackage{enumitem}
\usepackage{tikz}
\usetikzlibrary{positioning}

\usepackage{subcaption}

\usepackage{tabularx}
\newcolumntype{L}[1]{>{\raggedright\arraybackslash}p{#1}} % linksbündig mit Breitenangabe
\newcolumntype{C}[1]{>{\centering\arraybackslash}p{#1}} % zentriert mit Breitenangabe
\newcolumntype{R}[1]{>{\raggedleft\arraybackslash}p{#1}} % rechtsbündig mit Breitenangabe

\newcommand{\gqq}[1]{\glqq{}#1\grqq{}}
\newcommand{\gq}[1]{\glq{}#1\grq{}}
\newcommand{\dg}[1]{#1^\circ}

\renewcommand{\thesection}{Aufgabe \arabic{section}:}
\renewcommand{\thesubsection}{\alph{subsection})}
\renewcommand{\thesubsubsection}{\roman{subsubsection}}

\titleformat{\subsection}
{\normalsize}{\thesubsection}{1em}{}


\captionsetup[figure]{labelformat=empty}

\begin{document}

\title{Graphische Datenverarbeitung WS17/18 \\ Theorieübung 4}
\author{
  Salmah Ahmad (2880011)
  \and
  Markus Höhn (1683303)
  \and
  Tobias Mertz (2274355)
  \and
  Steven Lamarr Reynolds (1620638)
  \and
  Sascha Zenglein (2487032)
}

\maketitle

\section{Beleuchtung (4 Punkte)}

\subsection{(1 Punkt)}

\subsubsection{}

Umrechnung von RGB in HSV: \\
\newline
$
R' = R/255\\
G' = G/255\\
B' = B/255\\
max = max((R', G', B'))\\
min = min((R', G', B'))\\
\Delta = max - min
$\\~\\
$
H = \begin{cases}
0^\circ &\Delta = 0\\
60^\circ \times (\frac{G'-B'}{\Delta}) &, max = R'\\
60^\circ \times (\frac{B'-R'}{\Delta}+2) &, max = G'\\
60^\circ \times (\frac{R'-G'}{\Delta}+4) &, max = B'\\
\end{cases}
$\\~\\~\\
$
S = \begin{cases}
0 & max = 0\\
\frac{\Delta}{max}, &max \neq 0\\
\end{cases}
$\\~\\~\\
$
V = max
$\\

$
\text{Grün}_{HSV} = (120^\circ, 1, 1)
$\\
$
\text{Magenta}_{HSV} = (300^\circ, 1, 1)
$

\subsubsection{}

$
f_{RGB}(x) = ((1-x)\cdot255, \ x\cdot255, \ (1-x)\cdot255)
$\\
$
f_{HSV} = ((1-x)\cdot300^\circ + x\cdot 120^\circ, \ 1, \ 1)
$

\subsubsection{}

\subsubsection{}

\subsection{(1 Punkt)}

\subsection{(1 Punkt)}

\subsection{(0.5 Punkte)}

\subsection{(0.5 Punkte)}

\section{Texturierung (2 Punkte)}

\subsection{(1 Punkt)}

\subsection{(1 Punkt)} 

\section{Perspektivisch korrekte Texturierung (4 Punkte)}

\subsection{(1 Punkt)}

\subsection{(1 Punkt)} 

\subsection{(1 Punkt)}

\subsection{(1 Punkt)}


\end{document}
